\documentclass{discussion}

\usepackage{framed}
\usepackage[position=b]{subcaption}
\DeclareMathOperator{\Parents}{Parents}
\DeclareMathOperator{\NonDesc}{NonDesc}
\newcommand{\G}{\mathcal{G}}
\newcommand{\I}{\mathcal{I}}
\linespread{1.0}
\begin{document}

% Lecture Info
\lecture{9}{Explaining Away}{Benjamin R. Bray and Chansoo Lee}


%%%% INTRODUCTION %%%%%%%%%%%%%%%%%%%%%%%%%%%%%%%%%%%%%%%%%%%%%%%%%%%%%%%%%%%%%%%%%%%%%%%%%%%%%%
\section{Probabilistic influence}

Recall the HW5 Problem 3. 

\begin{lemma} 
\[P(t_1 | d_1) < P(t_1 | d_0)\]
\end{lemma}
Intuitively, this statement is kind of obvious. The person is on good diet $(d_1)$ should be less likely to test for high cholesterol $(t_1)$
\begin{proof}
From the factorization theorem, $T$ is independent of all variables other than $C$ given $D$. So, 
\begin{equation}
\label{eq:t_d_ind}
	P(T | D, u) = P(T | D)
\end{equation} for all values of $u$.

Now from the definition, we have that
\begin{equation}
	P(c_1 | d_1) < P(c_1 | d_0)
\end{equation}
meaning people with good diet is less likely to have a high cholesterol than those with bad diet. We were able to eliminate the conditioning on $u$ because of \eqref{eq:t_d_ind}.
Similarly,
\[P(t_1 | c_1) > P(t_1 | c_0).\]

We write
\[P(T | D) = \sum_{c} P(T, C = c | D) = \sum_{c} P(T| C = c) P(C = c | D)
\]
The second equality is because $T$ and $D$ are independent given $C$.

Now,
\[
P(t_1 | d_1) = P(t_1 | c_1) P(c_1 | d_1) +P(t_1 | c_0) P(c_0 | d_1)
\] 
and
\[P(t_1 | d_0) = P(t_1 | c_1) P(c_1 | d_0)
+P(t_1 | c_1) P(c_1 | d_0)
\]

We compute the difference:
\[P(t_1 | d_0) - P(t_1 | d_1) 
= P(t_1 | c_1) (P(c_1 | d_0) - P(c_1 | d_1))
+ P(t_1 | c_0) (P(c_0 | d_0) - P(c_0 | d_1))
\]
which is less than 0 because both terms in the parentheses are negative.
\end{proof}

\begin{corollary}
\[	P(t_1 | d_1) < P(t_1)\]
\end{corollary}
\begin{proof}
	Note that 
	\[P(t_1) = P(t_1 | d_1) p(d_1) + p(t_1 | d_0) p(d_0) < P(t_1 | d_1) p(d_1) + p(t_1 | d_1) p(d_0) = P(t_1 | d_1)\qedhere\]
\end{proof}

\begin{lemma}
\[	P(d_1 | e_1, m_1) < P(d_1 | m_1)\]
\end{lemma}
\begin{proof}
\[	P(d_1 | e_1, m_1)
 = P(m_1 | d_1, e_1) P(d_1) / P(m_1)
 > P(m_1 | d_1, e_0) P(d_1) / P(m_1)
 = P(d_1 | e_0, m_1).
\]
So,
\[P(d_1 | m_1) = P(d_1 | m_1, e_0) P(e_0) + P(d_1 | m_1, e_1)P(e_1) < P(d_1 | m_1, e_1) P(e_0) + P(d_1 | m_1, e_1)P(e_1) = P(d_1 | m_1,e_1)\]
\end{proof}

\section{Explaining Away}
The explaining away is the following phenomenon:
\begin{equation}
\label{eq:explainaway}
	P(h_1 | b_0, e_1) < P(h_1 | b_1, e_1).
\end{equation}
It follows that
\[P(h_1 | b_0, e_1) < P(h_1 | e_1)< P(h_1 | b_1, e_1).\]
Let's explore the sufficient and necessary condition for this to happen.

By the Bayes rule,
\[P(h_1 | e_1, B) = P(e_1 | h_1, B) \frac{P(h_1 | B)}{P(e_1 | B)} = P(e_1 | h_1, B) \frac{P(h_1)}{P(e_1 | B)}\]
since $H$ and $B$ are independent.


%  \eqref{eq:explainaway} is equivalent to 	
% \[\frac{P(e_1 | h_1, b_0)}{P(e_1 | b_0)} < \frac{P(e_1 | h_1, b_1)}{P(e_1 | b_1)}	\]
% which is also equivalent to
% \begin{equation}
% \label{eq:explainaway_ratio}
% \frac{P(e_1 | b_0)}{P(e_1 | h_1, b_0)} > \frac{P(e_1 | b_1)}{P(e_1 | h_1, b_1)}
% \end{equation}

Note that
\[P(e_1 | B) = P(e_1 | h_1, B)P(h_1) + P(e_1 | h_0, B)P(h_0)\]

So, 
\[\frac{1}{P(h_1|e_1,B)}
= 1 + \frac{P(e_1 | h_0, B)P(h_0)}{P(e_1|h_1,B)P(h_1)}\]

Hence, an equivalent statement of \eqref{eq:explainaway}
\[	1/P(h_1 | b_0, e_1) > 1/P(h_1 | b_1, e_1)
\]
is equivalent to 
\[1 + \frac{P(e_1 | h_0, b_0)P(h_0)}{P(e_1|h_1,b_0)P(h_1)} > 1 + \frac{P(e_1 | h_0, b_1)P(h_0)}{P(e_1|h_1,b_1)P(h_1)}
\]
which simplifies to
\[\frac{P(e_1 | h_0, b_0)}{P(e_1|h_1,b_0)} > \frac{P(e_1 | h_0, b_1)}{P(e_1|h_1,b_1)}.\]
% and similarly,
% \[P(e_1 | b_1) = P(e_1 | h_1, b_1)P(h_1) + P(e_1 | h_0, b_1)P(h_0)\]

Finally, we rearrange this:
\begin{equation}
\label{eq:explainaway_condition}
	\frac{P(e_1 | h_0, b_0)}{P(e_1 | h_0, b_1)} > \frac{P(e_1|h_1,b_0)}{P(e_1|h_1,b_1)}.
\end{equation}

So, the time constraint has more dramatic effect on the people that are not health-conscious. Health-conscious people are less likely to exercise if they are busy, but this probability drop is much less severe because they try to make time for exercise.

\begin{exercise}
	Give a realistic example where \eqref{eq:explainaway_condition} fails to hold.
\end{exercise}

\end{document}
